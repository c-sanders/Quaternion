%
% Document preamble
%

\documentclass[12pt, a4paper]{article}
\title{Quaternions.}
\author{Craig Sanders}
% \date{April 2025 onwards}
\setlength{\parindent}{0mm}

\usepackage{amsmath}
% Provides the * versions of \{equation} and \{align}

%
% Document body
%

\begin{document}

\maketitle

% ----------------------------------------
% Chapter 1)
% ----------------------------------------

\newpage

\section{What is a quaternion?}

Recall that a complex number has the following general form;

\begin{equation}\label{eqn:Complex_number_definition}
c = a + ib
\end{equation}

We say general form, because other people may choose to represent complex
numbers in a different manner.\\

We can clearly see that this complex number is comprised of two components; a
real part which is denoted by \(a\), and an imaginary part which is denoted by
\(ib\). These two components can be thought of as being mutually orthogonal to
one another.\\

But what would happen if we were to extend this complex number by adding more
imaginary components onto it? In particular, let's consider what would happen
if we were to add two more imaginary components onto it. In this case, the
number would now look as follows;

\begin{equation}\label{eqn:Quaternion_definition}
q = a + ib + jc + kd
\end{equation}

Since this number is comprised of multiple imaginary components, it belongs to
a class of numbers that is known as hyper-complex numbers. More specifically, it
is known as a quaterion since it is comprised of four components -- one real
and three imaginary, and ``quat'' comes from the Latin word for four. This is
the reason why the resulting number has now been assigned the letter `\(q\)',
because it is the starting letter of the word ``quaternion''.


% ----------------------------------------
% Chapter 2)
% ----------------------------------------

\newpage

\section{What a quaternion is not.}

But what does a quaternion mean or represent exactly, and what what can it be
used for?\\

Before we attempt to answer that, let us first recall 3-d Euclidean space. That
is, space which is often seen represented in mathematics textbooks with x, y and
z axes. Now consider the function;

\begin{equation}\label{eqn:0}
f(x,y,z) = \sqrt{x} + \sqrt{y} + \sqrt{z}
\end{equation}

and what will happen if we have the situation where;

\begin{equation*}
x=y=z=1
\end{equation*}

One might be inclined to think that the solution to this particular problem
would be;

\begin{equation}\label{eqn:1}
f(x,y,z) = i + j + k
\end{equation}

but this is not the case. Instead, the actual solution is;

\begin{align}
f(x,y,z) &= i + j + k \\
         &= i3
\end{align}

The intended purpose of all this, is to try and demonstrate that quaternions are
not -- repeat not, a 3-d imaginary number complement or equivalent of 3-d real
numbers, as we saw just a moment ago. Instead, the additional two imaginary
number axes which we added above, i.e. the \(j\) and \(k\) axes, represent even
higher order imaginary dimensions than what is represented by the \(i\)-axis.
Furthermore, we can't simply reach or get to these axes by taking the square
root of either the \(y\) or the \(z\) values, as we saw just above.\\

If the \(i\) axis represents numbers which reside in a higher order, domain, or
realm than the real numbers, then the \(j\) axis represents numbers in an even
higher order again; with the \(k\) axis taking it even one more step further.


% ----------------------------------------
% Chapter 2)
% ----------------------------------------

\newpage

\section{Multiplying two quaternions together.}

Imagine that we have two quaternions \(q_{1}\) and \(q_{2}\), which are defined
as follows;

\begin{align*}
q_{1} = a_{1} + ib_{1} + jc_{1} + kd_{1} \\
q_{2} = a_{2} + ib_{2} + jc_{2} + kd_{2}
\end{align*}

Multiplying the two of them together proceeds as follows;

\begin{equation*}
q_{1}q_{2} = (a_{1} + ib_{1} + jc_{1} + kd_{1})(a_{2} + ib_{2} + jc_{2} + kd_{2})
\end{equation*}

Expanding out the brackets yields;

\begin{align*}
q_{1}q_{2} = &a_{1}(a_{2}  + ib_{2} + jc_{2} + kd_{2}) + \\
             &ib_{1}(a_{2} + ib_{2} + jc_{2} + kd_{2}) + \\
             &jc_{1}(a_{2} + ib_{2} + jc_{2} + kd_{2}) + \\
             &kd_{1}(a_{2} + ib_{2} + jc_{2} + kd_{2})
\end{align*}

which becomes;

\begin{align*}
q_{1}q_{2} = &a_{1}a_{2}   + ia_{1}b_{2}     + ja_{1}c_{2}     + ka_{1}d_{2}  + \\
             &ib_{1}a_{2}  + i^{2}b_{1}b_{2} + ijb_{1}c_{2}    + ikb_{1}d_{2} + \\
             &jc_{1}a_{2}  + jic_{1}b_{2}    + j^{2}c_{1}c_{2} + jkc_{1}d_{2} + \\
             &kd_{1}a_{2}  + kid_{1}b_{2}    + kjd_{1}c_{2}    + k^{2}d_{1}d_{2}
\end{align*}

Applying the quaternion multiplication rules, i.e. \(ij = k\) and the
like, gives;

\begin{align*}
q_{1}q_{2} = &a_{1}a_{2}  + ia_{1}b_{2} + ja_{1}c_{2} + ka_{1}d_{2} + \\
             &ib_{1}a_{2} - b_{1}b_{2}  + kb_{1}c_{2} - jb_{1}d_{2} + \\
             &jc_{1}a_{2} - kc_{1}b_{2} - c_{1}c_{2}  + ic_{1}d_{2} + \\
             &kd_{1}a_{2} + jd_{1}b_{2} - id_{1}c_{2} - d_{1}d_{2}
\end{align*}

Grouping like terms together;

\begin{align*}
q_{1}q_{2} = &a_{1}a_{2}  - b_{1}b_{2}  - c_{1}c_{2}  - d_{1}d_{2}  + \\
             &ia_{1}b_{2} + ib_{1}a_{2} + ic_{1}d_{2} - id_{1}c_{2} + \\
             &ja_{1}c_{2} - jb_{1}d_{2} + jc_{1}a_{2} - jd_{1}b_{2} + \\
             &ka_{1}d_{2} + kb_{1}c_{2} - kc_{1}b_{2} + kd_{1}a_{2}
\end{align*}

Taking common factors out;

\begin{align*}
q_{1}q_{2} = &\bigl(a_{1}a_{2}  - b_{1}b_{2} - c_{1}c_{2} - d_{1}d_{2}\bigr) + \\
             &i\bigl(a_{1}b_{2} + b_{1}a_{2} + c_{1}d_{2} - d_{1}c_{2}\bigr) + \\
             &j\bigl(a_{1}c_{2} - b_{1}d_{2} + c_{1}a_{2} + d_{1}b_{2}\bigr) + \\
             &k\bigl(a_{1}d_{2} + b_{1}c_{2} - c_{1}b_{2} + d_{1}a_{2}\bigr)
\end{align*}

An alternate form for quaternion multiplication is as follows;

\begin{align*}
q_{1}q_{2} = a_{1}a_{2} - v_{1}v_{2},a_{1}v_{2} + a_{2}v_{1} + v_{1}\times v_{2}
\end{align*}

\end{document}